\begin{frame}{Konstrukcija pravokotnice na premico $p$ skozi točko $T$}
	\begin{columns}

		\column{0.55\textwidth}
		  \begin{itemize}
			 \item<1-> Dani sta premica $p$ in točka $T$.
			 \item<2-> Nariši lok $k$ s središčem v $T$.
			 \item<3-> Premico $p$ seče v točkah $A$ in $B$.
			 \item<4-> Nariši lok $m$ s središčem v $A$.
			 \item<5-> Nariši lok $n$ s središčem v $B$ in z enakim polmerom.
			 \item<6-> Loka se sečeta v točki $C$.
			 \item<7-> Premica skozi točki $T$ in $C$ je pravokotna na $p$.
		  \end{itemize}

		\column{0.45\textwidth}
		  \centering
		  \includegraphics<1>[width=50mm]{slike/fig-1.png}%
	      \includegraphics<2>[width=50mm]{slike/fig-2.png}%
		  \includegraphics<3>[width=50mm]{slike/fig-3.png}%
		  \includegraphics<4>[width=50mm]{slike/fig-4.png}%
		  \includegraphics<5>[width=50mm]{slike/fig-5.png}%
		  \includegraphics<6>[width=50mm]{slike/fig-6.png}%
		  \includegraphics<7>[width=50mm]{slike/fig-7.png}%
		
	\end{columns}

\end{frame}

				% Spodnje je za nalogo 3.4.

				% % Sliko smo naredili tako, da so točke A, B, T in C vse enako oddaljene
				% % od presečišča premic; kot ATC je 45°.
				% % Vsi krožni loki imajo radij 2.
				% \tikzmath{
				% 	% Razdalja od točke T do premice p je tako 2*sin(45°).
				% 	\t = 2*sin(45);
				% 	% Razdalja začetka loka m do premice p
				% 	% oz. razdalja točke T' levo in zgoraj od točke T do premice
				% 	\tt = 2*sin(60);
				% 	% Razdalja točke T' od navpične premice skozi T
				% 	\td = \t-2*cos(60);
				% }
				% % Definicija točke T
				% \coordinate [label={[blue, above left]:$T$}] (T) at (0,{\t});
				% % Risanje točke T
				% \fill[blue] (T) circle (2pt);
				% % Premica p
				% \draw[blue, very thick] (-2,0) -- (2,0) node[right] {$p$};
				% \pause
				% % Definicija pomožne točke A' in risanje krožnega loka k, ki se začne v A'
				% \coordinate (A') at ({-\tt},{\td});
				% \draw[gray, thin] (A') arc[start angle=210, end angle=330, radius=2] node[right] {\scriptsize $k$};
				% \pause
				% % Točka A
				% \coordinate [label=below left:{\scriptsize $A$}] (A) at ({-\t},0);
				% \draw (A) circle (1.5pt);
				
				% % Naloga 3.4.1.: Narišite še točko B (skupaj z oznako)
				
				% % Naloga 3.4.2.: Definirajte točko T', v kateri se začne lok m in narišite lok m z oznako.
				% % Lok je definiran s točko, v kateri se lok začne (ne središče!), z začetnim in končnim kotom ter radijem.
				% % Koti so vedno podani enako: kot 0 je v smeri x osi in se veča v nasprotni smeri urinega kazalca.

				% % Naloga 3.4.3.: Definirajte točko T'' in narišite lok n z oznako.
				
				% % Naloga 3.4.4.: Definirajte in narišite točko C.

				% % Naloga 3.4.5.: Narišite premico skozi točki T in C.

				% Konec vsebine za nalogo 3.4.


% Naloga 4
% \begin{frame}{Graf funkcije s TikZ}
% 	\centering
% 	\begin{tikzpicture}
% 		\begin{axis}[
% 			axis lines = middle,
% 			domain = 0:10,
% 			width = 9cm,
% 			height = 8cm,
% 			xtick = {0, 1, 2, 3, 4, 5, 6, 7, 8},
% 			ytick = {0, 1, 2, 3, 4, 5, 6},
% 			ymin = -1,
% 			ymax = 7,
% 			grid = both
% 		]
% 		\end{axis}
% 	\end{tikzpicture}
% \end{frame}